\section{A suitable computational task}
\label{sec:task}
To demonstrate quantum supremacy, we compare our quantum processor against ...

%%%%%%%%%%%%%%%%%%%%%%%%%%%%%%%%%%%%%%%%%%%%%%%%%%%%%%%%%%%%%%%%%%%%%%%%%%%%%%%%%%%%%%%%%%%%%%%%%
%
% This paragraph contains an example of math mode. This is an special and
% powerful mode that allows you to easily create math formulas and scientific
% environments. Use the special characters $ $ to enclose a math mode section.
% If you need it for your research paper, let me know and I will help you out.
%
% It also contains an example of italic format. Use the command \textit{•} to
% produce an italic format.
%
%%%%%%%%%%%%%%%%%%%%%%%%%%%%%%%%%%%%%%%%%%%%%%%%%%%%%%%%%%%%%%%%%%%%%%%%%%%%%%%%%%%%%%%%%%%%%%%%%
For a given circuit, we collect the measured bitstrings $\{x_{i}\}$ and compute
the \textit{linear cross--entropy benchmarking fidelity}, which is the mean of
the simulated probabilities of the bitstrings we measured:

%%%%%%%%%%%%%%%%%%%%%%%%%%%%%%%%%%%%%%%%%%%%%%%%%%%%%%%%%%%%%%%%%%%%%%%%%%%%%%%%%%%%%%%%%%%%%%%%%
%
% The following paragraph contains an example of an equation. Use the environments
% \begin{equation} \begin{split} ... \end{split} \end{equation} to create a numered
% multi-line equation. Each line is limited by the character '\\'. In order to align
% the equation you can use the character '&'.
%
% The equation within a research paper are identified by a number. This is done
% automatically by LaTeX. You have to define a \label{•} for future references.
%
%%%%%%%%%%%%%%%%%%%%%%%%%%%%%%%%%%%%%%%%%%%%%%%%%%%%%%%%%%%%%%%%%%%%%%%%%%%%%%%%%%%%%%%%%%%%%%%%%
\begin{equation}
  \begin{split}
    \mathcal{F}_{\text{XEB}} &= 2^{n} \braket{\Prob{x_{i}}}_{i} - 1, \\
  \end{split}
  \label{eq:fidelity}
\end{equation}

where $n$ is the number of ...
