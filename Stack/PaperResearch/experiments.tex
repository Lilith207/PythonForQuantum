\section{Modules and total design}
\label{sec:experiments}
Combining a GPU with the currently available quantum computing technique should give a vague description as to what a full QPU should look like. Keep in mind, that due to the fact that no data-sheets or exact hardware details on what these quantum chips would require and how they would communicate is available. Meaning that the design can only be up to a description of what components would be necessary and how it could work.
\\\\
First off the quantum chip which would be used. For the sake of making an assumption, it will be using 5 volts for most components. Together with a transformer to power the part which needs to generate a magnetic field at 24 volts.  This can be entirely off and is just an estimate. To keep the complexity relatively low, 5 of these smaller diamond chips will be used. The components to power this chip would be relatively common and not a huge hassle or cost. Besides of-course the diamond chip itself, which has to be produced in a lab under high pressure and heat. 
\\\\
There are several other components present on a GPU which will also be needed. These components are: VRM, PCIe, RAM, a PCB and some pin connectors. The VRM, PCB both somewhat speak for themselves, it needs to be able to distribute voltage around and the components should be soldered to something. The RAM is required as it needs to be able to both store results from running quantum circuits and to be able to store circuits it still needs to execute. To be able to take in the information of the circuits it needs to execute and to send back the results it will need a PCIe slot. This will also let the QPU fit into an everyday desktop setup. At last a set of pin connectors is needed, these will deliver the power to the QPU, it can be assumed that at max an 8-pin connector will be needed, although this might not be needed.
\\\\
It should be kept in mind that this would only give you a QPU. A device capable of speeding up a select set of classical problems via quantum methods. It would not be a replacement to a GPU and neither would it be a replacement to a CPU. Both of those would still be necessary, although they might not need to be as powerful when some tasks can be handed of to the QPU. This would not matter in the case of a desktop PC but could be vital in regards to a virtual reality headset or even a laptop, where weight and room come in short supply. Where a simple quantum chip would cost less space then the cooling system required for more powerful GPU and CPU's. 
\\\\
The full design would look akin to a regular GPU like in figure 1 as a lot of the component it needs are also needed for a QPU. Which would allow for a relatively setup for a new production line in companies like NVIDIA (or the production companies which manufacture for them). Seeing as the assembly lines to produce these GPU's are already present and it would only require some smaller modifications.