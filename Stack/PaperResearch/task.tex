\section{What would fit?}
\label{sec:task}
There are plenty of types of quantum computers out there. All with their upsides and downsides. Lets first take a look at what technologies are out there:

\begin{itemize}
  \item Superconducting
  \item Trapped ion
  \item Silicon dot
  \item Photonic
  \item Neutral atoms
  \item NV diamond
\end{itemize}

Where, to be able to fit one of these technologies into a QPU without too many assumptions, it would need to be able to function at room temperature and pressure. The choice already gets limited a lot here, as only NV diamonds can function at room temperature and pressure. This gives us the exact technology to be used for a QPU. 
\\\\
How does it work? The nv diamond quantum computing method is named after the NV diamond faults it uses. An NV fault is a type of fault which can both be naturally present in diamonds or be introduced. Where one carbon is replaced with a nitrogen and another connected carbon is completely removed, creating a vacancy. Which is where it gets its name from, Nitrogen Vacancy diamond faults \cite{wikipedia1}.
\\\\
These faults are then used together with a magnetic field or other method of delivering energy to steer and adjust the faults. So they can be operated on and measured. That being said, there is very little info on how this is done exactly out there, a lot of it is kept behind bars at the moment. So assumptions will have to be made regarding how this is to be controlled and how much voltage a Q-bit will need \cite{article1}.